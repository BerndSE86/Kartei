%%
%    * ----------------------------------------------------------------
%    * "THE BEER-WARE LICENSE" (Revision 42/023):
%    * Ronny Bergmann <mail@rbergmann.info> wrote this file. As long as
%    * you retain this notice you can do whatever you want with this
%    * stuff. If we meet some day and you think this stuff is worth it,
%    * you can buy me a beer or a coffee in return.
%    * ----------------------------------------------------------------
%
%
% A german example using the Kartei.cls - including print and toc as
% options, hence all pages are Din A4.
%
% Last Change: Kartei 1.9, 2012/01/04
%
\documentclass[a6paper
	,10pt
	,grid=none
	%,toc
	%,print
]{kartei}

\usepackage[utf8]{inputenc} %UTF8
%\usepackage{hyperref}
\usepackage{amsfonts}
\usepackage{amsmath}

\usepackage	[pdftex,
pdfauthor={Bernd Schulze Elfringhoff},
pdftitle={Karteikarten ANALYSIS},
pdfsubject={Karteikarten ANALYSIS}]{hyperref}


\begin{document}
  \setcardpagelayout
  \setlength{\parindent}{0em} 




%  \section{Analysis II}

%
%%
%%%%
%%%%%
%%%%%%
%%%%%%%
%%%%%%%%
%%%%%%%%%
%%%%%%%%%%
%%%%%%%%%% \subsection{Grundlagen und Wiederholung}
%%%%%%%%%%
%%%%%%%%%
%%%%%%%%
%%%%%%%
%%%%%%
%%%%%
%%%%
%%
%


\begin{karte}
	[SIN/COS Rechenregeln]
	{
		Wie lauten die Additionstheoreme für Sinus und Cosinus? 
		\newline
		\textit{Hinweis: $ \sin(a+b) = ... $}
	}
	[Basics]
	{
		$ \sin(a+b) = \sin a \cos b + \cos a \sin b$ \newline\newline
		$ \cos(a+b) = \cos a \cos b - \sin a \sin b$ \newline\newline
		$ \sin(a-b) = \sin a \cos b - \cos a \sin b$ \newline\newline
		$ \cos(a-b) = \cos a \cos b + \sin a \sin b$
	}
\end{karte}

\begin{karte}
	[Exp und Log]
	{
		(1) Wie lässt sich $log_{a}(x)$ durch ln ausdrücken?\newline\newline\newline
		(2) $log_a(x+y) = log_a(x)*log_a(y)$ - wahr oder falsch?\newline\newline\newline
		(3) $log_a(u^v) = ...$\newline\newline\newline
		(4) $log_a(u)-log_a(v) = ...$\newline\newline\newline
		(5) $ln(x)' = ...$\newline\newline\newline
		(6) $f:I\rightarrow \mathbb{R}, x\rightarrow f(x):=ln(x) $. Wie ist die Stammfunktion definiert?
	}
	[Basics]
	{
		(1) $log_{a}(x) = \frac{ln(x)}{ln(a)}$\newline\newline\newline
		(2) falsch, aber: $log_a(x \cdot y) = log_a(x)+log_a(y)$\newline\newline\newline
		(3) $log_a(u^v) = v \cdot log_a(u)$\newline\newline\newline
		(4) $log_a(u)-log_a(v) = log_a(\frac{u}{v})$\newline\newline\newline
		(5) $ln(x)' = \frac{1}{x}$\newline\newline\newline
		(6) $F:I\rightarrow \mathbb{R}, x\rightarrow F(x):=xln(x)-x +c $
	}
\end{karte}

\begin{karte}
	[Betrag - Dreiecksungleichung]
	{
		Beweise die Dreiecksungleichungen für den Betrag bei reellen Zahlen.
	}
	[Basics]
	{
		Dreiecksungleichung für reelle Zahlen: $\vert a+b \vert \leq \vert a \vert + \vert b \vert$ mit $a,b \in \mathbb{R}$
		\newline\newline
		$\vert a + b \vert = \sqrt{\vert a + b \vert^2} = \sqrt{\vert a^2 + 2ab + b^2 \vert} \leq \sqrt{\vert a^2 + \vert 2ab \vert + b^2 \vert} = \sqrt{(\vert a \vert + \vert b \vert )^2} = \vert a \vert + \vert b \vert$
	}
\end{karte}

\begin{karte}
	[Injektiv, Surjektiv, Bijektiv]
	{
		Sei $f:M\rightarrow N$ eine Funktion.
		\textit{(Zur Beantwortung der Fragen sollen möglichst nur Quantoren verwendet werden.)}
		\newline\newline
		Wie lautet die Definition für Injektivität? \newline\newline\newline\newline\newline
		Wie lautet die Definition für Surjektivität? \newline\newline\newline\newline\newline
		Wie lautet die Definition für Bijektivität? \newline\newline\newline\newline\newline
	}
	[Basics]
	{
		Sei $f:M\rightarrow N$ eine Funktion 
		\newline\newline
		\underline{Injektivität} \newline $\forall x_1, x_2 \in M: f(x_1)=f(x_2) \Rightarrow x_1=x_2$
		\newline\newline\newline
		\underline{Surjektivität} \newline $\forall y\in N \ \exists x \in M: y=f(x)$
		\newline\newline\newline
		\underline{Bijektivität} \newline
		f ist surjektiv und injektiv
	}
\end{karte}

\begin{karte}
	[Wichtige Reihen und deren Grenzwerte]
	{
		Wie sind die folgenden Reihen definiert, konvergieren diese und gegen welchen Grenzwert (falls sie konvergieren)?
		\newline\newline
		\underline{Harmonische Reihe:} \newline\newline\newline\newline
		\underline{Allgemeine harmonische Reihe:} \newline\newline\newline\newline
		\underline{Alternierende harmonische Reihe:} \newline\newline\newline\newline
		\underline{Geometrische Reihe:}
	}
	[Basics]
	{
		\underline{Harmonische Reihe}:\newline
		$\displaystyle \sum_{n=1}^{\infty} \frac{1}{n} = \infty$\newline\newline
		\underline{Allgemeine harmonische Reihe}:\newline
		$\displaystyle \sum_{n=1}^{\infty} \frac{1}{n^\alpha}$ konvergiert für $\alpha \in (1,\infty)$\newline\newline
		\underline{Alternierende harmonische Reihe}:\newline
		$\displaystyle \sum_{n=1}^{\infty} \frac{(-1)^{n+1}}{n} = ln(2)$\newline\newline
		\underline{Geometrische Reihe}:\newline
		$\displaystyle \sum_{n=0}^{\infty} x^n = \frac{1}{1-x}$ konvergiert für $|x|<1$
	}
\end{karte}

\begin{karte}
	[Wichtige Folgen und deren Grenzwerte]
	{
		Gegen welchen Grenzwert konvergieren die Folgen (falls sie konvergieren)?
		\newline\newline
		$\displaystyle \lim\limits_{k\rightarrow \infty} \frac{a}{n} = $
		\newline\newline
		$\displaystyle \lim\limits_{k\rightarrow \infty} a^n = $
		\newline\newline
		$\displaystyle \lim\limits_{k\rightarrow \infty} n^{\frac{1}{n}} = $ 
		\newline\newline
		$\displaystyle \lim\limits_{k\rightarrow \infty} \Big( 1+ \frac{1}{n} \Big)^n = $
		\newline\newline
		$\displaystyle \lim\limits_{k\rightarrow \infty} \Big( 1- \frac{1}{n} \Big)^n = $
		\newline\newline
		$\displaystyle \lim\limits_{k\rightarrow \infty} \Big( 1+ \frac{k}{n} \Big)^n = $
	}
	[Basics]
	{
		$\displaystyle \lim\limits_{k\rightarrow \infty} \frac{a}{n} = 0$ für $a\in\mathbb{R}$
		\newline\newline
		$\displaystyle \lim\limits_{k\rightarrow \infty} a^n = 0$ für $ \vert a \vert <1$
		\newline\newline
		$\displaystyle \lim\limits_{k\rightarrow \infty} n^{\frac{1}{n}} = 1$ für $a\in \mathbb{R}^+$
		\newline\newline
		$\displaystyle \lim\limits_{k\rightarrow \infty} \Big( 1+ \frac{1}{n} \Big)^n = e$
		\newline\newline
		$\displaystyle \lim\limits_{k\rightarrow \infty} \Big( 1- \frac{1}{n} \Big)^n = \frac{1}{e}$
		\newline\newline
		$\displaystyle \lim\limits_{k\rightarrow \infty} \Big( 1+ \frac{k}{n} \Big)^n = e^k$
	}
\end{karte}
%
%%
%%%%
%%%%%
%%%%%%
%%%%%%%
%%%%%%%%
%%%%%%%%%
%%%%%%%%%%
%%%%%%%%%% \subsection{Kurseinheit 1}
%%%%%%%%%%
%%%%%%%%%
%%%%%%%%
%%%%%%%
%%%%%%
%%%%%
%%%%
%%
%

\begin{karte}
	[Cauchy-Schwarz / Minkowski]
	{
		Wie lautet die Cauchy-Schwarz Ungleichung? \newline\newline\newline\newline\newline\newline\newline\newline\newline\newline
		Wie lautet die Minkowskische Ungleichung?
	}
	[KE1.1]
	{
		\underline{Cauchy-Schwarz Ungleichung:}
		\begin{equation}
			\begin{split}
				\vert \sum_{k=1}^{n} x_k y_k \vert &\leq \sqrt{\sum_{k=1}^{n}x^2_k} \sqrt{\sum_{k=1}^{n}y^2_k} \\
				\vert \langle x,y\rangle \vert^2 &\leq \langle x,x\rangle \cdot \langle y,y\rangle \\
				\vert \langle x,y\rangle \vert &\leq \Vert x \Vert \cdot \Vert y \Vert
			\notag\end{split}
		\end{equation}
		\underline{Minkowskische Ungleichung:}
		\begin{equation}
			\begin{split}
				\sqrt{\sum_{k=1}^{n} (x_k + y_k)^2} &\leq \sqrt{\sum_{k=1}^{n}x^2_k} \sqrt{\sum_{k=1}^{n}y^2_k} \\
				\Vert x+y \Vert &\leq \Vert x \Vert \cdot \Vert y \Vert
			\notag\end{split}
		\end{equation}
	}
\end{karte}

\begin{karte}
	[Vektorraum]
	{
		Wie ist ein reeller Vektorraum definiert
	}
	[KE1 - 1.3]
	{
		Sei X eine nicht leere Menge,  $x,y \in X, \alpha\in \mathbb{R}$. Dann heißt X ein reeller Vektorraum oder linearer Vektorraum über $\mathbb{R}$, wenn folgende Kriterien erfüllt sind \newline\newline
		\underline{Addition:} Auf X sei eine Addition erklärt, so dass x, y ein eindeutiges Element $x+y \in X$ zugeordnet ist mit:
		\newline
		- $Add_1$ (Assoziativität): $\forall x, y, z \in X: (x+y)+z =x+(y+z)$ \newline
		- $Add_2$ (Neutrales Element): $\exists \ (neutr.\ Element)\ 0 \in X$ mit $x+0=0+x=x\ \forall x\in X$\newline
		- $Add_3$ (Inverses): $\forall x\in X \ \exists (-x)\in X: x+(-x) =(-x)+x = 0\ (\forall x \in X)$\newline
		- $Add_4$ (Kommutativität): $\forall x, y \in X:\ x+y = y+x$
		\newline\newline
		\underline{Skalare Multiplikation:} Für $\alpha\in\mathbb{R}, x\in X$ ist eindeutig ein Element $\alpha x \in X$ zugeordnet mit:
		\newline
		- $Mult_1$ (Assoziativität): $\forall \alpha, \beta \in \mathbb{R}, x\in X: \ \alpha (\beta x) = (\alpha\beta)x$ \newline
		- $Mult_2$ (Distributivität): $\forall x, y \in X, \alpha, \beta \in \mathbb{R}: \ \alpha(x+y)=\alpha x + \alpha x,\ (\alpha + \beta)x=\alpha x + \beta x$
		- $Mult_3$ (Neutrales Element): $\forall x\in X: 1x=x$		
	}
\end{karte}

\begin{karte}
	[Normen - Äquivalenz]
	{
		Seien $\Vert \Vert$ und $\Vert \Vert^*$ zwei Normen auf $\mathbb{R}^n, a= ^t(a_1, ..., a_n)\in \mathbb{R}^n, U \subseteq \mathbb{R}^n, (x_k)_{k\in\mathbb{N}}$ mit $x_k = ^t(x_{k1}, ..., x_{kn})$ eine Folge in $\mathbb{R}^n$. Was kann man bzgl. der Äquivalenz der Normen in $\mathbb{R}^n$ hinsichtlich Umgebungsbegriff und Konvergenz sagen?
		\newline\newline
		(i) U ist Umgebung... 
		\newline\newline\newline\newline\newline\newline\newline\newline
		(ii)$(x_k)$ ist konvergent (gegen a) ...
	}
	[KE1.4]
	{
		Seien $\Vert\Vert$ und $\Vert\Vert^*$ zwei Normen auf $\mathbb{R}^n, a= ^t(a_1, ..., a_n)\in \mathbb{R}^n, U \subseteq \mathbb{R}^n, (x_k)_{k\in\mathbb{N}}$ mit $x_k = ^t(x_{k1}, ..., x_{kn})$ eine Folge in $\mathbb{R}^n$. Dann gilt:
		\newline\newline\newline
		(i) U ist Umgebung von a in $(\mathbb{R}^n, \Vert\Vert) \Leftrightarrow$ U ist Umgebung von a in $(\mathbb{R}^n, \Vert\Vert^*)$ 
		\newline\newline\newline\newline\newline\newline\newline\newline
		(ii)$(x_k)$ ist konvergent (gegen a) in $(\mathbb{R}^n, \Vert\Vert)$ \newline
		$\Leftrightarrow (x_k)$ ist konvergent (gegen a) in $(\mathbb{R}^n, \Vert\Vert^*)$ \newline
		$\Leftrightarrow$ Für jedes $j\in 1, ...n$ konvergiert $(x_{kj})$ (gegen $a_j$) in $(\mathbb{R}, \vert \vert)$ \newline
	}
\end{karte}

\begin{karte}
	[Normen]
	{
		Wie sind die folgenden Normen (auf $\mathbb{R}^n$) definiert?
		\newline\newline
		\underline{Durch $d_1(x,y)$ induzierte Norm:}		
		\newline\newline\newline\newline\newline\newline
		\underline{Euklidische Norm:}
		\newline\newline\newline\newline\newline\newline
		\underline{Maximumnorm:}
	}
	[KE1.4 - 1.4.4]
	{
		\underline{Durch $d_1(x,y)$ induzierte Norm:}
		\newline
		$\displaystyle \Vert x \Vert_1 := \left( \sum_{k=1}^{n} \vert x_k \vert \right)$
		\newline\newline\newline
		\underline{Euklidische Norm:}
		\newline
		$\displaystyle \Vert x \Vert_2 := \sqrt{\left( \sum_{k=1}^{n} \vert x_k \vert^2 \right)}$
		\newline\newline\newline
		\underline{Maximumnorm:}
		\newline
		$\displaystyle \Vert x \Vert_\infty := \max \{\vert x_k\vert \ \vert 1\leq k \leq n \}$
	}
\end{karte}

\begin{karte}
	[Konvergenz]
	{
		Wie lautet die Definition für eine Cauchyfolge? \\ (in einem normierten Raum)
	}
	[KE1.5 - 1.5.14]
	{
		Sei $(X,\Vert \Vert)$ ein normierter Raum. Eine Folge $(x_k)$ in X heißt Cauchyfolge in $(X,\Vert \Vert)$, wenn die folgende Bedingung erfüllt ist: 
		\newline\newline\newline\newline
		\underline{Alternative 1:}\newline
		$ \forall \epsilon>0 \ \exists n_0 \in \mathbb{N} \ \forall k \in \mathbb{N} \ : (k>n_0 \Rightarrow \Vert x_k - x_{n_o} \Vert < \epsilon )$
		\newline\newline\newline\newline
		\underline{Alternative 2:}\newline
		$ \forall \epsilon>0 \ \exists n_0 \in \mathbb{N} \ \forall k,l \in \mathbb{N} \ : (k,l>n_0 \Rightarrow \Vert x_k - x_l \Vert < \epsilon )$
	}
\end{karte}

\begin{karte}
	[Hausdorffeigenschaft]
	{
		Wie ist die Hausdorfeigenschaft definiert (\textit{für normierte Räume})?
	}
	[KE1.5 - 1.5.3]
	{
		Sei $(X,\Vert)$ ein normierter Raum, und seien $a,b \in X$ mit $a \neq b$. Dann gibt es eine Umgebung U von a und eine Umgebung V von b mit $U \cap V = \emptyset$
	}
\end{karte}

\begin{karte}
	[Konvergenz in $(X,\Vert \Vert)$]
	{
		Sei $(X,\Vert \Vert)$ ein normierter Raum. Welche der Aussagen ist wahr?
		\newline \newline
		(1) Jede konvergente Folge ist eine Cauchyfolge \newline\newline
		(2) Jede Cauchyfolge ist beschränkt \newline\newline
		(3) Jede Cauchyfolge konvergiert \newline\newline
		(4) Besitzt eine Cauchyfolge eine konvergente Teilfolge, so ist sie selbst konvergent
	}
	[KE1.5 - 1.5.15]
	{
		(1) wahr \newline\newline 
		(2) wahr \newline\newline
		(3) wahr für Banachräume (wie bspw. $(\mathbb{R}^n, \Vert \Vert)$)\newline\newline
		(4) wahr
	}
\end{karte}

\begin{karte}
	[Normen]
	{
		Seien $(\mathbb{R},\Vert \Vert)$ und $(\mathbb{R},\Vert \Vert_\infty)$ normierte reelle Räume mit einer beliebigen Norm $\Vert \Vert$ und der Maximumnorm $\Vert \ \Vert_\infty$.
		Seien $a = \textsuperscript{t}(a_1,...,a_n) \in \mathbb{R}^n$, 
		$(x_k)_{k\in \mathbb{N}}$ mit $\textsuperscript{t}(x_{k1},..x_{kn})$ eine Folge in $\mathbb{R}^n$
		\newline\newline
		(1) Welcher Zusammenhang besteht zwischen $\Vert\Vert$ und $\Vert\Vert_\infty$ \newline\newline\newline\newline
		(2) Hängen Umgebungen zwischen $(\mathbb{R},\Vert \Vert)$ und $(\mathbb{R},\Vert \Vert_\infty)$ zusammen?\newline\newline\newline\newline
		(3) Welche drei Konvergenzen sind hier äquivalent? (in drei verschiedenen normierten Räumen)
	}
	[KE1.5]
	{
		(1) $\Vert x\Vert \leq \alpha \Vert x\Vert_\infty$ and $\Vert x\Vert_\infty \leq \beta \Vert x\Vert$ für jedes $x\in\mathbb{R}$
		\newline\newline\newline\newline
		(2) U ist Umgebung von a in $(\mathbb{R},\Vert \Vert) \Leftrightarrow$  U ist Umgebung von a in $(\mathbb{R},\Vert \Vert_\infty)$ 
		\newline\newline\newline\newline
		(3) $(x_k)$ konvergiert gegen a in $ (\mathbb{R},\Vert \Vert) \Leftrightarrow $ $(x_k)$ konvergiert gegen a in $ (\mathbb{R},\Vert \Vert_\infty) \Leftrightarrow $ Für jedes v = 1,...,n konvergiert $(x_{kv})_{k\in\mathbb{N}}$ gegen $a_v$ in $(R,||)$
	}
\end{karte}

\begin{karte}
	[Metrik]
	{
		Wie ist eine Metrik $d(x,y)$ definiert?
	}
	[KE1.4.SE]
	{
		Sei X eine beliebige Menge. Eine Abbildung $d : X \times X \rightarrow \mathbb{R}$ heißt Metrik auf X, wenn für beliebige Elemente $x, y, z \in X$ die folgenden Axiome erfüllt sind:
		\newline\newline\newline
		(1) Positive Definitheit: 	$d(x,y) \geq 0$ und $d(x,y)=0 \Leftrightarrow x=y$
		\newline\newline\newline
		(2) Symmetrie: 	$d(x,y)=d(y,x)$
		\newline\newline\newline
		(3) Dreiecksungleichung: $d(x,y) \leq d(x,z)+d(z,y)$ 
	}
\end{karte}

\begin{karte}
	[Norm]
	{
		Auf dem Vektorraum $C^1([a,b])$ der stetig differenzierbaren Funktionen auf [a,b] wird definiert:
		$ \Vert \ \Vert : C^1 ([a,b]) \rightarrow \mathbb{R}, f \rightarrow \Vert f \Vert := \Vert f \Vert_\infty + \Vert f' \Vert_\infty$ \newline\newline
		Zeige, dass $\Vert \ \Vert$ eine Norm ist.
	}
	[KE1.4.SE]
	{
		Auf dem Vektorraum $C^1([a,b])$ der stetig differenzierbaren Funktionen auf [a,b] wird definiert:
		$ \Vert \ \Vert : C^1 ([a,b]) \rightarrow \mathbb{R}, f \rightarrow \Vert f \Vert := \Vert f \Vert_\infty + \Vert f' \Vert_\infty$ \newline\newline
		\underline{(i) Definitheit $\Vert x \Vert \geq 0$ und $(\Vert x \Vert = 0 \Leftrightarrow x=0)$:}
		\newline
		$\Vert f \Vert = \Vert f \Vert_\infty + \Vert f' \Vert_\infty \geq 0$
		\newline\newline\newline
		\underline{(ii) positive Homogenität $\Vert \alpha x \Vert$ = $\vert \alpha \vert \cdot \Vert x \Vert$:}
		\newline
		$\Vert \alpha f \Vert
		= \Vert \alpha f \Vert_\infty + \Vert \alpha f' \Vert_\infty 
		= \vert \alpha \vert (\Vert f \Vert_\infty + \Vert f' \Vert_\infty)
		= \vert \alpha \vert \Vert  f \Vert_\infty
		$
		\newline\newline\newline
		\underline{(iii): Dreiecksungleichung $\Vert x +y \Vert \leq \Vert x \Vert + \Vert y \Vert$:}
		\newline
		$\Vert f+g \Vert
		= \Vert f+g \Vert_\infty + \Vert f'+g' \Vert_\infty 
		\leq (\Vert f \Vert_\infty + \Vert g \Vert_\infty) + (\Vert f' \Vert_\infty + \Vert g' \Vert_\infty)
		= (\Vert f \Vert_\infty + \Vert f' \Vert_\infty) + (\Vert g \Vert_\infty + \Vert g' \Vert_\infty)
		=\Vert f \Vert + \Vert g \Vert
		$
	}
\end{karte}

\begin{karte}
	[Metrik]
	{
		Sei M eine nichtleere Menge. Zeige, dass durch $d_d := \scriptsize \begin{cases} 0 & \text{falls }p=q \\ 1 & \text{falls }p \neq q \normalsize \end{cases}$ eine Metrik auf M definiert wird, die diskrete Metrik.
	}
	[KE1.4 - UniMarburg]
	{
		Sei M eine nichtleere Menge. Sei $d_d := \scriptsize \begin{cases} 0 & \text{falls }p=q \\ 1 & \text{falls }p \neq q \normalsize \end{cases}$. Dann ist $d_d$ eine Metrik.
		\newline\newline\newline
		\underline{(1) Positive Definitheit: 	$d(x,y) \geq 0$ und $d(x,y)=0 \Leftrightarrow x=y$}
		\newline
		$d(p,q) \geq 0$ und $d(p,q)=0 \Leftrightarrow p=q$ ist nach Definition gegeben.
		\newline\newline
		\underline{(2) Symmetrie: 	$d(x,y)=d(y,x)$}
		\newline
		Sei $d(p,q)=0$, also $p=q \Rightarrow$ $d(x,y)=0=d(y,x)$
		\newline
		Sei $d(p,q) \neq 0$, also $p \neq q \Rightarrow$ $d(x,y)=1=d(y,x)$
		\newline\newline
		\underline{(3) Dreiecksungleichung: $d(x,y) \leq d(x,z)+d(z,y)$ }
		\newline
		Falls $p=q: d(p,q) = 0 \leq d(p,r) + d(r,q)$ \newline
		Falls $p \neq q: d(p,q) = 1 \leq d(p,r) + d(r,q)$, da mind. $p \neq r$ oder $r \neq q$ sein muss.
	}
\end{karte}

\begin{karte}
	[Abgeschlossenheit - Hausdorff]
	{
		Sei $(M,d)$ ein metrischer Raum. Zeige, dass für alle $x \in M$ die Menge $\{x\} \subseteq M$ abgeschlossen ist unter Verwendung der Hausdorffeigenschaft.
	}
	[KE1.5.SE]
	{
		Sei $x \in \{x\} \subseteq M$ entsprechend Aufgabenstellung.
		\newline\newline
		Sei $y \in M \setminus \{x\}$ beliebig. Wegen der Hausdorffeigenschaft des Raums existieren Umgebungen $U_x$ von x, $U_y$ von y mit $U_x \cap U_y = \emptyset$. Da $U_y$ Umgebung von y ist, gibt es auch eine offene Menge $U^*_y$ mit $U^*_y \subseteq U_y$
		\newline\newline
		(i) Für jedes $y \in M \setminus \{x\}$ gilt $x \notin U^*_y$, also $U^*_y \subseteq M \setminus \{x\}$. Da dies für jedes $y \in M \setminus \{x\}$ gilt, gilt zudem auch $\bigcup\limits_{y\in M \setminus \{x\}} U^*_y \subseteq M \setminus \{x\}$
		\newline\newline
		(ii) Andererseits muss auch gelten $M \setminus \{x\} \subseteq\bigcup\limits_{y\in M \setminus \{x\}} U^*_y$, da wir um jedes $y\in M \setminus \{x\}$ eine Umgebung legen. M muss dann natürlich in diesen gesamten Umgebungen liegen.
		\newline\newline\newline
		Aus (i) und (ii) folgt $M \setminus \{x\} = \bigcup\limits_{y\in M \setminus \{x\}} U^*_y$. Als Vereinigung offener Mengen ist $M \setminus \{x\}$ damit auch offen. Damit muss $\{x\}$ aber abgeschlossen sein.
	} 
\end{karte}



%
%  \subsection{VORLAGE}
%

\begin{karte}
	[Thema]
	{
		...
	}
	[Kursabschnitt]
	{
		...
	}
\end{karte}

\end{document}