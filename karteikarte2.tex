%
%		* ----------------------------------------------------------------------------
%		* "THE BEER-WARE LICENSE" (Revision 42/023):
%		* Ronny Bergmann <mail@darkmoonwolf.de> wrote this file. As long as you retain 
%		* this notice you can do whatever you want with this stuff. If we meet some day,
%		* and you think  this stuff is worth it, you can buy me a beer or a coffee in return. 
%		* ----------------------------------------------------------------------------
%
%
% Beispiel zur Dokumentvorlage für Din A6 Karteikarten 
% -- Version 1.6 --
%
\documentclass[print,a6paper]{kartei}

\usepackage[utf8]{inputenc} %UTF8
\usepackage[OT1]{fontenc}
\usepackage[scaled]{helvet}
\usepackage[ngerman]{babel} % Neue Rechtschreibung
%Menge an Paketen die für vor allem mathematisches und informatisches gut brauchbar ist
\usepackage{graphicx,textcomp,booktabs,mathptmx,courier,calc,trfsigns}
\usepackage{enumerate,subfigure,listings,color,amsmath,amssymb,euscript}
%
% Deutsche Absatzformatierung
%
\setlength{\parindent}{0pt} 
\setlength{\parskip}{1em}	

\begin{document}
	
	\begin{karte}{Lebensphilosophie}{Prüfungsfrage}{Wie lautet die Antwort auf die Frage nach dem Leben dem Universum und dem Ganzen Rest}
	42
	\end{karte}
	
	\begin{karte}{Zahlenkunde}{}{Was ist der Unterschied in der Verwendung von Drölf und $n$ bei „Ihnen“ ?}
	$n$ wird verwendet für Zahlen bis hin zu „verdammt groß“, Drölf nur bis hin zu verdammt.
	\end{karte}

	\begin{karte}{Informatik}{LabelKarte}{Was ist verschränkte Rekursion ?}
	\label{karte:antwort} Siehe Antwort auf Karte \ref{karte:frage}
	\end{karte}

	\begin{karte}{}{}{Was ist die Antwort auf Karte \ref{karte:antwort} ?}
		\label{karte:frage} Siehe Karte \ref{karte:antwort}
		
		Und der Absatz ? 
	\end{karte}
\end{document}