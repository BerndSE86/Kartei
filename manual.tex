\documentclass[a4paper]{article}
\usepackage[a4paper]{geometry}
\usepackage[utf8]{inputenc} %UTF8
\usepackage[OT1]{fontenc}
\usepackage[scaled]{helvet}
\usepackage[ngerman]{babel} % Neue Rechtschreibung
%Menge an Paketen die für vor allem mathematisches und informatisches gut brauchbar ist
\usepackage{graphicx,hyperref}
\usepackage{enumerate,subfigure,listings}
\lstloadlanguages{TeX}
\lstset{language={},frame=none,numbers=left,basicstyle=\sffamily,breaklines=true,tabsize=3, showspaces=false, showtabs=false, columns=fixed}	
\setlength{\parindent}{0pt} 
\setlength{\parskip}{1em}	

\setlength{\marginparwidth}{2cm}
\begin{document}
\title{Karteikarten in \LaTeX\\{\normalsize Version 1.7}}
\author{Ronny Bergmann\\\texttt{mail@darkmoonwolf.de}}
\date{November 2008}
\maketitle
\section{Einleitung}
Um Karteikarten in \LaTeX\ zu verfassen genügen zunächst die einfachen Formate Din A6, A7 und A8. Möchte man jedoch das Ergebnis auch drucken, so muss man die Reihenfolge umsortieren. Dazu bietet dieses Paket neben einer Kartenumgebung für die Definition der Karteikarten ein Druckformat.

\section{Dokumentoptionen}
Die normalen Optionen eines Artikels, wie etwa Schriftgröße (Standard hier auf \lstinline!10pt!), können selbst auch übergeben werden und werden dann 
weitergegeben. Die Karteikarten sind stets im Querformat gesetzt.
\subsection{Kartenformat}
Es gibt die folgenden Karteikartenformate
\paragraph{\lstinline!a6paper!} Im Din A6-Karteikarten-Format $(148$mm$\times 105$mm$)$ sind die Ränder ein wenig größer gewählt, als bei den anderen beiden Formaten. Dieses Format ist der Standard, wenn nichts angegeben wird für das Format.
\paragraph{\lstinline!a7paper!} Din A7-Karteikarten $(105$mm$\times 74$mm$)$
\paragraph{\lstinline!a8paper!} Din A8-Karteikarten $(74$mm$\times 52$mm$)$, diese scheinen recht klein, hier wirkt die Standardschriftgröße fast etwas groß.
\subsection{Druck}
\paragraph{\lstinline!print!} Die Option \lstinline!print! in der Dokumentklasse schaltet zwischen der Setzart der Karten als einzelne Seiten und dem Satz auf eine DinA4-Seite um.

Auf der Rückseite der Karteikarten wird zusätzlich ein Rand um jede Karteikarte erzeugt, der beim Ausschneiden helfen soll.

Die Anordnung sind 4 (bei A6), 8 (bei Din A7) bzw. 16 (bei Din A8) Karten auf einer Seite. Dabei sind A6 und A8 auf Querformat gesetzt, A7 auf Hochformat. Dadurch muss beim Duplex-Druck Für A6 und A8 \emph{über die kurze Seite geklappt} gedruckt werden, bei A7 \emph{über die lange Seite}.

Die Karteikarten werden in ihrer Originalgröße auf dem Din-A4-Blatt angeordnet, zum Druck sollte man die Option „Auf Seitengröße verkleinern“ des jeweiligen Druck-Dialogs wählen. Dann sind die Karten um den Druckrand verkleinert.
\subsection{Beispiele}
\begin{itemize}
	\item Zum Erstellen von A6-Karteikarten, normal gesetzt \lstinline!\documentclass[a6paper]{kartei}!
	\item Für A7-Karteikarten im Druckformat \lstinline!\documentclass[a7paper, print]{kartei}!
\end{itemize}
%
%
%
\section{Die Kartenumgebung}
Die Kartenumgebung heißt \lstinline!karte!. Die Karten werden automatisch durchnummeriert und es ist möglich, per \lstinline!\ref{}! auf Karten zu verweisen, die \lstinline!\label{}! enthalten, dann wir an der Stelle automatisch die Karteikartennummer eingesetzt, etwa „\# 3“. Die Umgebung benötigt 3 Parameter.
\begin{enumerate}
	\item[optional] Dem Fach oder wesentlichen Stichwort
	\item[Pflicht] Der Frage oder dem Titel der Karteikarte
	\item[optional] Dem Kommentar oder zweitem Stichwort
\end{enumerate}
In der Umgebung selbst wird dann die eigentliche Antwort angegeben. Diese wird auf die Rückseite der Karteikarte gesetzt.

Die Vorderseite enthält im Kopf links das Fach, mittig die Nummer der Karte, rechts den Kommentar. Zentral auf der Vorderseite wird die Frage gesetzt. Auf der Rückseite wird links die Kartennummer wiederholt, mittig der Term „Antwort“.

Die beiden Stichworte Fach und Kommentar lassen sich auch global festlegen, ebenso wie der Antwortterm sich verändern lässt. Details dazu finden sich im Abschnitt \ref{subsec:Struktur}.

\newpage

\subsection{Beispiele}
\begin{itemize}
	\item exemplarisch gefüllte Karteikarte
	\begin{lstlisting}
		\begin{karte}
			[Fach]
			{Frage oder Titel}
			[Kommentar]
		Antwort
		\end{karte}
		
	\end{lstlisting} 
	\item Karteikarte mit einer wichtigen Frage
	\begin{lstlisting}
		\begin{karte}
			[Lebensphilosophie]
			{Wie lautet die Antwort auf die Frage nach dem Leben dem Universum und dem Ganzen Rest?}
			[wichtig!]
			42
		\end{karte}
		
	\end{lstlisting}
\end{itemize}
%\fach{standardfach}
%\fach{} zum expliziten leeren
%\fachstil{ --stilangabe-- }
%\kommentar{standardfach}
%\kommentar{} zum expliziten leeren
%\kommentarstil{ --stilangabe-- }
%\antwort{Antwort}
%\antwortstil{ --stilangabe-- }

\subsection{Strukturierung}\label{subsec:Struktur}
\paragraph{Fächer \& Kommentare}
Neben der Möglichkeit bei einer Karte das Fach und den Kommentar explizit anzugeben lässt sich beides auch global setzen, so dass es bei den darauffolgenden Karten verwendet wird.

Mit \lstinline!\fach! (engl. \lstinline!\subject!) gefolgt von einem Text in \lstinline!{geschweiften Klammern}! setzt man das Fach für die nachfolgenden Karten. Die Schriftformatierung kann über Neudefinition des Befehl \lstinline!\fachstil! (engl. \lstinline!\subjectstyle!) vorgenommen werden. Standardmäßig ist der Stil auf \emph{kursiv} gesetzt. Um an einer Stelle den Momentanen Wert Auszugeben, gibt es den Befehl \lstinline!\dasfach{}!. Wird bei einer Karte bei gesetztem Fach trotzdem ein Fach angegeben, so hat das Fach der Karte Vorrang, so kann in einem großen Block auch eine Ausnahmekarte erzeugt werden.

Analog lässt sich der Kommentar global setzen mit \lstinline!\kommentar{Kommentartext}! (engl. \lstinline!\comment!), dessen Stil mit \lstinline!\kommentarstil! (engl. \lstinline!\commentstyle!), respektive die Ausgabe mit \lstinline!\derkommentar! (engl. \lstinline!\thecomment!). Auch dies wird von einem lokalen Wert, der bei einer Karte angegeben wird überschrieben, so dass in einem Block von Karten mit gleichem Kommentar auch eine einzelne Ausnahme angegeben werden kann

Um also für das Fach „Lebensphilosophie“ eine Reihe von Karten zu erstellen, wobei eben jenes Fach in \emph{kursiv} gesetzt sein soll, benötigt man also
\begin{lstlisting}
	\fach{Lebensphilosophie}
	\renewcommand{\fachstil}{\emph}
\end{lstlisting}
Direkt vor der ersten Karteikarte bei der dies wirksam sein soll. Alle darauf folgenden Karten ohne Angabe des optionalen Fach-Parameters werden mit dem Fach Lebensphilosophie ausgegeben.

\paragraph{Antworttext auf der Rückseite}\label{par:Antwort} Und wiederum nochmals Analog lässt sich der Antworttext setzen mittels \lstinline!\antwort! (engl. \lstinline!\answer!) bzw. dessen Stil über Neudefinition von \lstinline!\antwortstil! (engl. \lstinline!\answerstyle!). Zusätzlich ist auch der Antworttext im Fließtext wiedergebbar mittels \lstinline!\dieantwort! (engl. \lstinline!\theanswer!, und dieser Befehl gibt nicht ausschließlich „42“ aus).

Um also den Antworttext auf Esperanto anzugeben, also auf „respondo“ zusetzen, was gleichzeitig in \textsc{Kapitälchen} gesetzt werden soll, verwendet man die beiden Befehle
\begin{lstlisting}
	\antwort{respondo}
	\renewcommand{\antwortstil}{\textsc}
\end{lstlisting}

\paragraph{Section \& Subsection} Zusätzlich kann man eine automatische Nummerierung der Fächer vornehmen, indem man diese mittels \lstinline!\section! diese Fächer setzt. Möchte man die Nummerierung für ein Fach zwischendrin aussetzen, so kann man \lstinline!\section*! verwenden. Dies läuft analog mit den Kommentaren und \lstinline!\subsection! bzw. für ein setzen des Kommentars ohne Nummerierung \lstinline!\subsection*!.

\subsection{Kartennummerierung}
Die Nummerierung der Karten ist standardmäßig definiert mit
\begin{lstlisting}
	\renewcommand{\theCardID}{\emph{\# \arabic{CardID}}}
\end{lstlisting}
Also einem führenden \# gefolgt von der Nummer der Karte. Diese Anzeige wird auf der Vorder- und Rückseite der Karte sowie bei Verweisen verwendet. Durch ändern des Befehlt \lstinline!\theCardID! kann dies beeinflusst werden.
%
%
%
\section{technische Details}
\subsection{Benötigte Pakete}
Um dieses Kartenpaket zu verwenden muss man mindestens die Pakete \lstinline!fancyhdr!, \lstinline!vmargin!, \lstinline!xargs!, \lstinline!pgfpages! und \lstinline!tiks! installiert haben. Diese finden sich alle im CTAN.
\subsection{Die Kartenumgebung}
Die Kartenumgebung beruht auf der \lstinline!twoside!-Variante des \lstinline!article!, und setzt damit die oben beschriebenen Sachen im Kopf für jede Karteikarte. Der Zähler wird dabei inkrementiert und als Standard vorher gesetzt.
\begin{lstlisting}[title=Die Kartenumgebung]
\newenvironmentx{karte}[3][1=\card@fach,3=\card@kommentar]
{
	\pagestyle{fancy}
	{
		\fancyhead[LO]{\dasfach{#1}} %\emph{#1}} %Fach
		\fancyhead[RO]{\derkommentar{#3}} %Kommentar
	}
	\pagestyle{fancy}{% 
	}%
~\vfill{~\hfill \parbox[t]{.9\textwidth}{\centering \Large #2}\hfill~}\vfill~
\refstepcounter{CardID}
\newpage%
}
{
	\fancyhead[LO]{\emph{\textbf{Achtung:} Antwortseite der Karte ist zu voll}} %vorlaeufige Loesung
	\cleardoublepage
}
\end{lstlisting}
\subsection{Das Drucklayout}
Das Drucklayout ist eine Variation der durch \lstinline!pgfpages! aus dem Projekt \lstinline!pgf & tikz!\footnote{\url{http://sourceforge.net/projects/pgf/}} bereitgestellten Befehle zur Anordnung mehrerer Seiten / Präsentationsfolien auf einer einzigen Seite. Diese wurdn modifiziert, so dass alle geraden Seiten (Vorderseiten) auf der ersten und alle ungeraden Seiten (Rückseiten) auf der Rückseite angeordnet werden. Die Rückseite ist außerdem in der Reihenfolge so verändert, dass das Ergebnis Duplex-Druckbar ist. Die Ränder auf der Rückseite werden ebenso damit erzeugt. 
%
%
%
\section{bekannte Probleme \& weitere Ideen}
\paragraph{„Inhalt der Rückseite zu umfangreich“} % (fold)
Ist die Antwort zu lang oder zu umfangreich, so wird eine neue Seite begonnen, das führt dazu, dass eine an sich als Vorderseite gedachte Seite den zweiten Teil der Antwort enthält. Hier ist die Kopfzeile mit einem Hinweis versehen. Die darauffolgende Karteikarte beginnt wieder auf einer ungeraden Seite. Die Nummerierung ist davon nicht betroffen. Gelöst wird dies vorherst durch ein \lstinline!~\cleardoublepage!, 
% _inhalt_der_rückseite_zu_groß_ (end)

\paragraph{Positionierung von Kartennumer, Fach \& Kommentar festlegen}
Eine Erweiterungsidee ist, dass man selbst die Positionierung der Elemente im Kopf festlegen kann und diese bei Bedarf auch in die Fußzeile legen kann, etwa den Kommentar in die Fußzeile der Vorderseite.

\paragraph{Rückseitenformat festlegen} 
Für die Rückseite könnte man noch ein Format festlegen. Dies könnte die Wahl umfassen, ob normale Absätze vorhanden sein sollen, oder nicht. Außerdem wären Felder denkbar zum Eintragen, wie gut man eine Karteikarte beherrscht.

\paragraph{Liste der Karteikarten \textbackslash Lernkontrolltabelle} 
Im Druck-Modus könnte man mit einem Index der Karteikarten sowohl eine Themenübersicht als auch eine Lernkontrolle einbinden, auf der man seine Lernfortschirtte verzeichnen kann.
\newpage
\section{Lizenz}
\begin{lstlisting}[basicstyle=\footnotesize\sffamily, numbers=none]
*
* -----------------------------------------------------------------------------------
* "THE BEER-WARE LICENSE" (Revision 42/023):
* Ronny Bergmann <mail@darkmoonwolf.de> wrote this file. As long as you retain 
* this notice you can do whatever you want with this stuff. If we meet some day,
* and you think  this stuff is worth it, you can buy me a beer or a coffee in return. 
* -----------------------------------------------------------------------------------
*
\end{lstlisting}

\section{Changelog}
\begin{description}
	\item[1.7 - 02.11.2008] Fach \& Kommentartext sind jetzt global setzbar über \lstinline!\section! und \lstinline!\subsection! (und deren \lstinline!*!-Derivate), Druckoptionen entfernt., Antworttext veränderbar
	\item[1.6 - 09.09.2008] A7, A8-Karteikarten und die Druckränder eingefügt. Erste Version mit diesem Manual
\end{description}
\end{document}
