\documentclass[a4paper]{article}
\usepackage[a4paper]{geometry}
\usepackage[utf8]{inputenc} %UTF8
\usepackage[OT1]{fontenc}
\usepackage[scaled]{helvet}
\usepackage[ngerman]{babel} % Neue Rechtschreibung
%Menge an Paketen die für vor allem mathematisches und informatisches gut brauchbar ist
\usepackage{graphicx,hyperref}
\usepackage{enumerate,subfigure,listings}
\lstloadlanguages{TeX}
\lstset{language={},frame=none,numbers=left,basicstyle=\sffamily,breaklines=true,tabsize=3, showspaces=false, showtabs=false, columns=fixed}	
\setlength{\parindent}{0pt} 
\setlength{\parskip}{1em}	

\setlength{\marginparwidth}{3cm}
\newcommand\leftpar[1]{\marginpar[]{\sffamily #1}}
\begin{document}
\title{Karteikarten in \LaTeX}
\author{Ronny Bergmann\\\texttt{mail@darkmoonwolf.de}}
\date{September 2008}
\maketitle
\section{Einleitung}
Um Karteikarten in \LaTeX\ zu verfassen genügen zunächst die einfachen Formate Din A6, A7 und A8. Möchte man jedoch das Ergebnis auch drucken, so muss man die Reihenfolge umsortieren. Dazu bietet dieses Paket neben einer Kartenumgebung die Definition der Karteikarten an sich und des Druckformates.

\section{Dokumentoptionen}
\leftpar{\textbackslash documentclass}
Die normalen Optionen eines Artikels, wie etwa Schriftgröße (Standard hier auf \lstinline!10pt!), können selbst auch übergeben werden und werden dann 
weitergegeben. Die Karteikarten sind stets im Querformat gesetzt.
\subsection{Kartenformat}
Es gibt die folgenden Karteikartenformate
\paragraph{\lstinline!a6paper!} Im Din A6-Karteikarten-Format $(148$mm$\times 105$mm$)$ sind die Ränder ein wenig größer gewählt, als bei den anderen beiden Formaten. Dieses Format ist der Standard, wenn nichts angegeben wird für das Format.
\paragraph{\lstinline!a7paper!} Din A7-Karteikarten $(105$mm$\times 74$mm$)$
\paragraph{\lstinline!a8paper!} Din A8-Karteikarten $(74$mm$\times 52$mm$)$, diese scheinen recht klein, hier wirkt die Standardschriftgröße fast etwas groß.
\subsection{Druck}
\paragraph{\lstinline!print!} Die Option lstinline!print! in der Dokumentklasse schaltet zwischen der Setzart der Karten als einzelne Seiten und dem Satz auf eine DinA4-Seite um.

Auf der Rückseite der Karteikarten wird zusätzlich ein Rand um jede Karteikarte erzeugt, der beim Ausschneiden helfen soll.

Die Anordnung sind 4 (bei A6), 8 (bei Din A7) bzw. 16 (bei Din A8) Karten auf einer Seite. Dabei sind A6 und A8 auf Querformat gesetzt, A7 auf Hochformat. Dadurch muss beim Duplex-Druck Für A6 und A8 \emph{über die kurze Seite geklappt} gedruckt werden, bei A7 \emph{über die lange Seite}.
\subsection{Druck - Randoptionen}
Die Randoptionen sind nur wirksam, wenn die Option \lstinline!print! angegeben ist. Dann gibt es die folgenden Möglichkeiten. 
\paragraph{\lstinline!noborder!} Ergibt einen Satz ohne Rand. Das bedeutet, dass die Karteikarten im Druck exakt die Größe haben, die sie nach dem Din-A-Standard haben müßten. Allerdings ist dieses PDF nur druckbar, wenn der Drucker randlos drucken kann, ansonsten fehlen einige Zeichen am Rand.
\paragraph{\lstinline!smallborder!} Hier wird ein relativ kleiner Rand um die Karten dadurch erzeugt, dass diese auf 98\% verkleinert werden. Damit ist das Format in einigen Druckern zwar immer noch nicht druckbar, es wird auf der anderen Seite aber nur ein klein wenig verkleinert. 
\paragraph{\lstinline!bigborder!} Hier werden die Karten auf 96\% verkleinert. Damit sollte das Ergebnis auf fast allen Druckern ohne Abschnitte druckbar sein.

Alternativ kann auch die \lstinline!noborder!-Variante vor dem Druck entsprechend verkleinert werden mit der Option „automatische Seitenskalierung“ verkleinert werden. Die Variante ohne Rand wird gesetzt, wenn keine Randoption angegeben wird.

\subsection{Beispiele}
\begin{itemize}
	\item A6-Karteikarten, normal gesetzt \lstinline!\documentclass[a6paper]{kartei}!
	\item A7-Karteikarten, Druckformat mit großem Rand\\ \lstinline!\documentclass[a7paper, print, bigborder]{kartei}!
\end{itemize}
%
%
%
\section{Die Kartenumgebung}
Die Kartenumgebung heißt \lstinline!karte!. Die Karten werden automatisch durchnummeriert und es ist möglich, per \lstinline!\ref{}! auf Karten zu verweisen, die \lstinline!\label{}! enthalten, dann wir an der Stelle automatisch die Karteikartennummer eingesetzt, etwa „\# 3“. Die Umgebung benötigt 3 Parameter.
\begin{enumerate}
	\item[optional] Dem Fach oder wesentlichen Stichwort
	\item[Pflicht] Der Frage oder dem Titel der Karteikarte
	\item[optional] Dem Kommentar oder zweitem Stichwort
\end{enumerate}
In der Umgebung selbst ist die Antwort bzw. Rückseite der Karteikarte.

Die Vorderseite enthält im Kopf links das Fach, mittig die Nummer der Karte, rechts den Kommentar. Zentral auf der Vorderseite wird die Frage gesetzt. Auf der Rückseite wird links die Kartennummer wiederholt, mittig der Term „Antwort“. Hier wird schließlich der Inhalt der Umgebung wiedergegeben.

\subsection{Beispiele}
\begin{itemize}
	\item exemplarisch gefüllte Karteikarte
	\begin{lstlisting}
		\begin{karte}
			[Fach]
			{Frage oder Titel}
			[Kommentar]
		Antwort
		\end{karte}
		
	\end{lstlisting} 
	\item Karteikarte mit einer wichtigen Frage
	\begin{lstlisting}
		\begin{karte}
			[Lebensphilosophie]
			{Wie lautet die Antwort auf die Frage nach dem Leben dem Universum und dem Ganzen Rest?}
			[wichtig!]
			42
		\end{karte}
		
	\end{lstlisting}
\end{itemize}
%\fach{standardfach}
%\fach{} zum expliziten leeren
%\fachstil{ --stilangabe-- }
%\kommentar{standardfach}
%\kommentar{} zum expliziten leeren
%\kommentarstil{ --stilangabe-- }
%\antworttext{\textsc{Antwort}}

\subsection{Strukturierung}
\paragraph{Fächer \& Kommentare}
Sowohl das Fach als auch der Kommentar können fest besetzt werden, damit man das Fach nicht immer neu angeben muss. Dies erfolgt mit dem Befehl \lstinline!\fach{}!, ist das Fach damit auf einen nichtleeren Wert gesetzt, wird der in der \lstinline!karte!-Umgebung übergebene Text ignoriert und das Fach ausgegeben. Analog verläuft es mit dem Kommentar über den Befehl \lstinline!\kommentar{}!. Standardmäßig sind beide auf leer gesetzt, so dass die Parameter der Kartenumgebung ausgegeben werden. 
Zusätzlich läßt sich mit \lstinline!\fachstil{}! und \lstinline!\kommentarstil{}! eine Formatierung des Textes angeben. Möglich sind dabei die üblichen \LaTeX-Anweisungen wie etwa \lstinline!\textbf!, \lstinline!\textsc!, \lstinline!\emph!,$\ldots$ Diese Formatierung betrifft auch die normal als Parameter übergebenen Fächer und Kommentare der nachfolgenden Karteikarten. 

Um also für das Fach „Lebensphilosophie“ eine Reihe von Karten zu erstellen, wobei eben jenes Fach in \emph{kursiv} gesetzt sein soll, benötigt man also
\begin{lstlisting}
	\fach{Lebensphilosophie}
	\fachstil{\emph}
\end{lstlisting}
Direkt vor der ersten Karteikarte bei der dies wirksam sein soll.

Außerdem kann auch der zentral auf der Rückseite platzierte Wert (Standardmäßig steht dort „\emph{Antwort}“) mit dem Befehl \lstinline!\antworttext{}! verändert werden. Eventuelle Formatierungen dieses Eintrags sind auch innerhalb des Befehls anzugeben. Der Standard ist also
\begin{lstlisting}
	\antworttext{\emph{Antwort}}
\end{lstlisting}

\paragraph{Section \& Subsection} Zusätzlich kann man eine automatische Nummerierung der Fächer vornehmen, indem man diese mittels \lstinline!\section! diese Fächer setzt. Möchte man die Nummerirung für ein Fach zwischendrin aussetzen, so kann man \lstinline!\section*! verwenden. Dies läuft analog mit den Kommentaren und \lstinline!\subsection! bzw. für ein setzen des Kommentars ohne Nummerierung \lstinline!\subsection*!
%
%
%
\section{technische Details}
\subsection{Die Kartenumgebung}
Die Kartenumgebung beruht auf der \lstinline!twoside!-Variante des \lstinline!article!, und setzt damit die oben beschriebenen Sachen im Kopf für jede Karteikarte. Der Zähler wird dabei inkrementiert und als Standard vorher gesetzt.
\begin{lstlisting}[title=Die Kartenumgebung]
\newenvironment{karte}[3]
	{% Vor der Umgebung: Vorderseite bauen
		\pagestyle{fancy}{% 
			\fancyhead[LO]{\emph{#1}} %Fach
			\fancyhead[RO]{\emph{#2}} %Kommentar
		}%
		~\vfill{~\hfill \parbox[t]{.9\textwidth}{\centering \Large #3}\hfill~}\vfill~
		\refstepcounter{CardID}
		\newpage%
	}%
	{% Nach Umgebung: Warnung im Kopf (bis der Umbruch verhindert wird) 
	 % und neue Karteikarte auf ungerader Seite beginnen lassen
		\fancyhead[LO]{\emph{\textbf{Achtung:} Rueckseite der Karte ist zu voll}}
		\cleardoublepage
	}
\end{lstlisting}
\subsection{Das Drucklayout}
Das Drucklayout ist eine Variation der durch \lstinline!pgfpages! aus dem Projekt \lstinline!pgf & tikz!\footnote{\url{http://sourceforge.net/projects/pgf/}} bereitgestellten Befehle zur Anordnung mehrerer Seiten / Präsentationsfolien auf einer einzigen Seite. Diese wurdn modifiziert, so dass alle geraden Seiten (Vorderseiten) auf der ersten und alle ungeraden Seiten (Rückseiten) auf der Rückseite angeordnet werden. Die Rückseite ist außerdem in der Reihenfolge so verändert, dass das Ergebnis Duplex-Druckbar ist. Die Ränder auf der Rückseite werden ebenso damit erzeugt. 
%
%
%
\section{bekannte Probleme \& weitere Ideen}
\paragraph{„Inhalt der Rückseite zu umfangreich“} % (fold)
Ist die Antwort zu lang oder zu umfangreich, so wird eine neue Seite begonnen, das führt dazu, dass eine an sich als Vorderseite gedachte Seite den zweiten Teil der Antwort enthält. Hier ist die Kopfzeile mit einem Hinweis versehen. Die darauffolgende Karteikarte beginnt wieder auf einer ungeraden Seite. Die Nummerierung ist davon nicht betroffen. Gelöst wird dies vorherst durch ein \lstinline!~\cleardoublepage!, 
% _inhalt_der_rückseite_zu_groß_ (end)
\paragraph{selbst festlegen, welche Felder in Kopf und Fuß was enthalten, etwa unten rechts das Fach o.ä.}
\paragraph{Rückseitenformat festlegen}
\newpage
\section{Lizenz}
\begin{lstlisting}[basicstyle=\footnotesize\sffamily, numbers=none]
*
* -----------------------------------------------------------------------------------
* "THE BEER-WARE LICENSE" (Revision 42/023):
* Ronny Bergmann <mail@darkmoonwolf.de> wrote this file. As long as you retain 
* this notice you can do whatever you want with this stuff. If we meet some day,
* and you think  this stuff is worth it, you can buy me a beer or a coffee in return. 
* -----------------------------------------------------------------------------------
*
\end{lstlisting}

\section{Changelog}
\begin{description}
	\item[1.7 - TODO] Fach \& Kommentartext global definierbar, Antworttext veränderbar
	\item[1.6 - 09.09.2008] A7, A8-Karteikarten und die Druckränder eingefügt. Erste Version mit diesem Manual
\end{description}
\end{document}
